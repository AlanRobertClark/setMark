\documentclass{ArcHandout}
\usepackage[widetext]{A4ee}
\usepackage{ArcMacro}
\title{Guidelines on Setting and Marking of Examination
       Papers.\\\vspace*{1em}\hrule}

\begin{document}
\makeheads

%%tth: \href{SettingMarkingExams.pdf}{Printable pdf version}



\begin{abstract}
The following is a summary of key points to bear in mind in setting
and marking examinations. The details pertaining to the cover sheet of the
examination paper must be adhered to at all times. The various criteria
relating to the format and content of the examination paper should be
applied as far as possible. If an exam paper departs from these guidelines
it must be subjected to critical scrutiny and justification.
\end{abstract}

\section{Overall strategy}
\begin{enumerate}
  \item The questions set in the paper must be in accordance with the
  Course Brief \& Outline. An important test is whether an independent
  observer can get a good idea of the course objectives by studying the
  question paper.

  \item The paper should be constituted to contain:
    \begin{itemize}
      \item a reasonable proportion of straightforward basic material (say
      30\%) 

      \item a body of material of moderate complexity (say 50\%)

      \item a limited amount of more demanding work (say 20\%) that,
      perhaps, only those students that have mastered the concepts covered
      in the course material very well can be expected to complete.  (This
      could increase for exams where the total marks available exceed 100\%
      (see section \ref{sec:format}).)

      \item The paper should not contain more than 40\% bookwork, seen
      problems, or theory.
    \end{itemize}

  \item Wherever possible, questions should be designed to be marked in a
  limited time. The need for infinitely detailed working should be avoided
  and questions should focus on examining principles rather than
  manipulative dexterity.
\end{enumerate}

\section{Estimating the time required for a question}

\begin{enumerate}
  \item A guideline used by experienced lecturers is that in a normal
  three-hour paper where five questions must be done, the examiner should
  be able to work out a complete solution for one of the questions and write
  it out in reasonable detail in 15 to 20 minutes.  The solution should
  take about 1 or 1.5 sides of A4 paper. This does not apply to descriptive
  or essay type questions where a limit of 300-400 words per full question
  is advised for any course in the engineering curriculum.

  \item Questions should be posed concisely, and as a guide, if a question
  takes more than 100 words to pose, it should be looked at critically.
  This may differ depending on the subject.
\end{enumerate}

\section{Sub-sections in a question}
\begin{enumerate}
  \item There should not be too many sub-sections in a question (suggested
  maximum is 3).

  \item It is inadvisable to assign too low a mark to each section
  (suggested minimum 3/20 (or 3\% of total)).

  \item Sub-sections of a question should not be dependent on earlier
  sub-sections. Where there are dependencies, these should be grouped
  together under one sub-section with one global mark for that sub-section.
  This gives the examiner more flexibility in assessing how the student
  went about the solution. The mark breakdown for the sub-sections of a
  question should be given.

  \item The proportion of the paper which depends strongly on the outcome
  of a previous section, must be strictly limited or preferably avoided
  altogether.
\end{enumerate}

\section{Use of tutorial questions and/or past examination questions}
\begin{enumerate}

  \item It is normal practice to use tutorial questions or past examination
  questions to illustrate the standard. It is important that the students
  should have had tutorial questions of up to and possibly beyond full
  examination standard. 
  
  \item A briefing session before the examination should tell the students
  what preparation is important for the particular exam.  Any relevant
  information that the examiner feels in important (e.g. what data would be
  available in examination) should be made known during this session.

\end{enumerate}

\section{Format of paper}
\label{sec:format}
\begin{enumerate}
  \item For a 3 hour examination, no choice; 5 out of 6 questions; or 5 out
  7 questions are all acceptable formats. In the case of no choice, there
  should be more than 100 marks (100\%) available. As a guide, setting a
  paper out of 120 marks is recommended in this case.  Various other
  permutations are also acceptable (e.g. 4 out of 5)---in particular the
  case where 20\% of the mark may be compulsory and allocated to a
  multiple-choice question.  This also applies to PG courses. 

  \item It is advisable to be consistent from year to year, and students
  \emph{must be informed} of any changes in format.

  \item if open book (i.e. specified requirements and/or
  restrictions) exams are set, it should be made very clear to the
  students that the aim is only to avoid us having to provide extensive
  tables or data.  The standard or level of the questions should not be any
  higher than for normal closed book exams.

  \item Where information is provided to students (e.g. tables, formulas,
  charts) the student should have seen the material during the course.
  Information should not be too voluminous. If it exceeds 4 sides of A4 it
  should be examined critically. (If it is voluminous this might suggest an
  open book exam.)

  \item It is usually not advisable to have a single question counting too
  high a proportion of the paper (say 40\%) This is particularly true if it
  is probable that some students will get it completely wrong. 
  \end{enumerate}

\section{Calculators}
\begin{enumerate}
  \item Examiners should bear in mind that programmable calculators are now
  allowed in all closed book examinations. These papers should be
  set so that students without ``fancy''  calculators are not unduly
  disadvantaged. If this is not acceptable, an open-book exam must be specified
  at the start of the course.

  \item It is the understanding that ``school-type'' non-programmable
  scientific calculators can be specified up to second year. (Therefore not
  needing an A4 information sheet.) However, third-year and above should
  not have this restriction (unless NO calculators are specified), implying
  a mandatory A4 information sheet.

  \item Standard school policy on the A4 information sheet:

  An A4 information sheet may be brought into the examination. Both sides
  of the sheet may be used for text, figures and equations, but it must be
  hand-written. No printed or photostatic copies are allowed. No additional
  reading aids are allowed. 

  \item Any further retrictions to the A4 sheet (eg text and equations
  only) etc \emph{MUST} be communicated to the student via the course brief
  and outline.
  
\end{enumerate}  


\section{Layout}
\begin{enumerate}

  \item Diagrams should be placed within or immediately after the questions
  to which they apply.

  \item The page header should contain the course code and course name,
  and the page number in the style ``page 1 of 4'' to flag misprints.

  \item Any blank pages (eg to separate Smith Charts) must be flagged
  ``This page intentionally blank''.

  \item All examination papers must include the standard University front
  page (which is automatically produced by the standard School LaTeX{}
  \href{http://ytdp.ee.wits.ac.za/ArcExam.zip}{exam class.}) The front page
  of the examination paper must include at least the following information:
  \begin{itemize}
    \item {course or topic no(s)}--- e.g. ELEN 422
    \item {course topic name(s)}---e.g. High Voltage Engineering
    \item {examination date}---e.g. July 2001
    \item {year of study}---e.g. fourth year
    \item {degree for which the examination is prescribed}---e.g.  B.Sc.
    (Eng.)
    \item {faculty(ies) presenting candidates}---e.g. Engineering
    \item {internal examiner(s) and telephone extension numbers}---e.g. Dr.
    AB See x1234
    \item {external examiner(s)}---e.g. Prof IC Nothing
    \item {special material required}---e.g. graph paper, brain
    \item {time allowed}
    \item {special instructions.} Note that these must state that examiners
    require student to show all working in their papers, and to submit all
    relevant scripts.
  \end{itemize}
\end{enumerate}

\section{Procedures for Setting Examination Papers}

\begin{enumerate}
  \item A typed first version, complete with diagrams and solution
  (draft-quality), shall be prepared in time for the school workshop on
  setting examinations in each term.

  \item After the workshop, revisions are made and the paper is forwarded
  to the external examiner for return by the specified deadline. Solutions
  and the course brief and outline must be included.

  \item The paper is finalized in the light of the external examiner's
  comments. The Head of School will arbitrate in cases of difficulty.

  \item The internal examiner is responsible for signing the approval on
  the inside of the cover page.

  \item The paper is forwarded to the School administration, who will keep
  a copy in the safe, and forward a copy to the examinations office by hand
  and a receipt (or signature) must be obtained.

  \item At all stages when the paper and solutions are not actually being
  worked on, they must be locked up in a safe cabinet/cupboard.
\end{enumerate}

\section{Marking of Examination Scripts}
\begin{enumerate}

  \item The School Administration will provide an Excel spreadsheet of the
  registered students for the course, obtained from the SIRS system, which
  must be used for the recording of test and examination marks. These must
  be forwarded to the School Administration \emph{immediately} after
  marking, for backup purposes.

  \item No marking or recording of marks must be done for any student
  \emph{not} on the SIRS system.

  \item In the third and fourth year of study, two External Examiners
  oversee the student's performances for all examinations for that
  particlar year. They also go though scripts of borderline students and
  others chosen at random.

  \item The other aspect is that of checking. The School employs checkers
  who go through and ensure that each question has been marked, and that
  the totals are correct. This has proved to be an important part of the
  procedure, and detects many errors, which both Internal and the External
  examiners have missed. This procedure does require that some formality be
  brought into our methods of marking.  The following marking scheme, must
  be applied:

  \begin{enumerate}
    \item Each page that has been critically looked at should have a red
    line (please avoid using the same colour pen as the student has) down
    the right-hand margin. Alternatively, it must be clear that the work on
    the page has been marked (e.g. by means of a mark allocation, ticking
    of the work etc.) This indicates to the checker that that part of the
    script has been marked.

    \item All pages on which there is any writing at all should have the
    red line down the right-hand margin so as to indicate to the checker
    that this page has been looked at. Intervening blank pages should have
    a diagonal red line drawn through them.

    \item Where a question is divided into sections, the individual marks
    awarded for each section should be written at the right-hand margin.
    The total number of marks for any question must be ringed, in the
    right-hand margin, at the end of the question.

    \item Where a question is incomplete and continued at the later stage,
    an arrow should be placed at the break point in the question to
    indicate to the checker that additional marking for that question can
    be found later in the script. The later section should be clearly
    labeled to facilitate easy checking.

    \item The total marks awarded per question should then be transferred
    to the outside over. The final total mark should then be inserted at
    the bottom of the marking grid on the front page of the script.

    \item Multiple choice marks must be entered on the front page of the
    script by the internal examiner.
  \end{enumerate}
  
  \item Test scripts
  \begin{enumerate}
    \item All of the above considerations apply. 
    \item It should be remembered that test scripts are returned to
    students. The method of marking should preclude alterations and
    additions by students after the paper has been returned. Material
    written in pencil or altered with Tippex must not under any
    circumstances be re-discussed with the student.
    \item It must be made clear to the student before the test that pencil
    may be used, but that it cannot be ``re-marked''.
    \item Any question answered in pencil must be identified as such by
    writing ``pencil'' down the left hand margin of the script during
    marking.
    \item The free space in the script must be marked off---e.g. using
    a diagonal red line. 
  \end{enumerate}
\end{enumerate}

\section{Checklist for Examinations}
\begin{tabular}{|p{5mm}|p{0.9\textwidth}|}\hline
&Does the question paper reflect the scope and objectives of the
  course?\\\hline
&Does the paper contain: a proportion of straightforward basic material
(30\%)\\
&Does the paper contain: material of moderate complexity (50\%)\\
&Does the paper contain: some more demanding questions (20\%)\\\hline
&Are the questions posed concisely?\\\hline
&Is the supporting data longer than 4 pages for a ``closed book''
  exam?\\\hline
&Is the mark allocation shown? \\\hline
&Does any one mark allocated count more than 20\%?\\\hline
&Does any one mark allocated count less than 3\%?\\\hline
&Is any section of a question strongly dependent on a previous
  section?\\\hline
&Is the amount of choice adequate? (75\%--80\% of paper need be
  done to score 100\%)\\\hline
&Are the front page and the layout of the paper in accordance with
  School recommendations? \\\hline
&If the format does not follow that of previous
  years, have the students been informed?\\\hline
\end{tabular}

\section{Checklist for Class Tests}
\begin{tabular}{|p{5mm}|p{0.9\textwidth}|}\hline
  & Does the question paper reflect the work done over the period that
  is being tested in lectures and tutorials?\\\hline

  & Does the paper contain:
     a proportion of straightforward basic material (30\%)\\
  & Does the paper contain:
     material of moderate complexity (50\%)\\
  & Does the paper contain:
     some more demanding questions (20\%)\\\hline

  & Is the supporting data longer than 4 pages for a ``closed book''
  exam?\\\hline

  & Is the mark allocation shown?\\\hline

  & Does any one mark allocated count more than 50\%?\\\hline

  & Does any one mark allocated count less than 5\%?\\\hline

  & Is any section of a question strongly dependent on a previous
  section?\\\hline

  & Is the layout of the paper such that it makes it easy to read and
  digest?\\\hline
\end{tabular}

\end{document}
